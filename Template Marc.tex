\documentclass[a4paper, 11pt]{article}

%
% PACKAGES
%

\usepackage[ngerman]{babel} % Rechtschreibprüfung
\usepackage[utf8]{inputenc} % Enkodierung von Umlauten
\usepackage{fancyhdr} % Konfiguration des Headers und Footers
\usepackage{graphicx} % Einbindung von Grafiken
\usepackage{listings} % Codelistings
\usepackage{pdfpages} % Einbingung externer PDF Dokumente
\usepackage{geometry} % Konfiguration der Seitenränder
\usepackage[hidelinks]{hyperref} % Entfernung der Hyperlinkmarkierung durch rote Ränder
\usepackage{hyperref} % Einbindung von Hyperlinks
\usepackage{setspace} % Konfiguration des Zeilenabstands
\usepackage{verbatim} % Für Blockkommentare

\usepackage{blindtext}

%
% SETUP
%

% Package: listings
% Konfiguration des Codelistings
% Default: Java als Programmiersprache, graue Schrift auf gelbem Hintergrund

\lstset{columns=fullflexible,
        mathescape=true,
        literate=
               {-=-}{$\rightarrow{}$}{1}
               {-==-}{$\tab{}$}{1},
        morekeywords={if,then,else,return}
        }
        
\lstset{language=Java,
  showspaces=false,
  showtabs=false,
  numbers=left,
  breaklines=true,
  showstringspaces=false,
  breakatwhitespace=true,
  commentstyle=\color{pgreen},
  keywordstyle=\color{pblue},
  backgroundcolor=\color{isabelline},
  captionpos=b,
  frame=tlrb
  frameround=rrrr,
  stringstyle=\color{pred},
  basicstyle=\ttfamily,
}

% Package: geometry
% Anpassung der Seitenränder an vorgegebene Richtlinien
% Default: Kopf: 3,5 cm; Fuß: 2,5 cm; Links: 3,5 cm; Rechts: 2,5 cm

\geometry{verbose,a4paper,tmargin=35mm,bmargin=25mm,lmargin=35mm,rmargin=25mm}

% Package: fancyhdr
% Anpassung der Kopf- und Fußzeile
% Default: Kopf (rechts): Aktuelles Kapitel, Kopf (links): Autor, Fuß (rechts): Seitenzahl

\pagestyle{fancy}

\fancyhf{}
\rhead{Nachname, Vorname}
\lhead{\leftmark}
\rfoot{\thepage}

% Einrückungen
\setlength{\parskip}{0pt}
\setlength\parindent{20pt}

\onehalfspacing % 1.5 Zeilenabstand

\bibliographystyle{ieeetr} % Quellenverzeichnis nach IEEE Standard

% Beginn des Dokuments
\begin{document}

\includepdf[pages={1}]{\detokenize{Template_Titelseite.pdf}} % Einbindung der Titelseite

% Erklärung

\section*{Erklärung}
\label{sec:erklaerung}
\addcontentsline{toc}{section}{\nameref{sec:erklaerung}}

Hiermit erkläre ich, dass dieser Bericht auf meinen eigenen Leistungen beruht. Insbesondere erkläre ich, dass
\begin{description}
	\item [a)] ich diesen Praktikumsbericht selbstständig ohne unzulässige fremde Hilfe erstellt habe,
	\item [b)] die Verwendung aller Quellen klar und korrekt angegeben habe und aus anderen Quellen entnommene Zitate eindeutig als solche gekennzeichnet habe,
	\item [c)] ich aus anderen Quellen entnommene Gedanken, Ideen, Bilder, Zeic
	\item [d)] ich außer den angegebenen Quellen und Hilfsmitteln keine weiteren Quellen und Hilfsmittel zur Erstellung dieses Berichts verwendet habe.
\end{description}

\begin{verbatim}




\end{verbatim}

\begin{tabular}{lp{2em}l} 
 \hspace{5cm}   && \hspace{4cm} \\\cline{1-1}\cline{3-3} 
 Ort, Datum     && Unterschrift 
\end{tabular}

\newpage

% Falls die wissenschaftliche Arbeit ein Praxisbericht ist

\section*{Sperrvermerk}
\label{sec:sperrvermerk}
\addcontentsline{toc}{section}{\nameref{sec:sperrvermerk}}

Dieser Bericht enthält vertrauliche Informationen der Firma $<$Firma$>$. Eine Veröffentlichung oder Vervielfältigung dieses Berichts außerhalb der Verwendung zur Bewertung und Dokumentation der Studienleistungen ohne ausdrückliche Genehmigung der Firma $<$Firma$>$ ist nicht gestattet.
Dieser Bericht ist nur Personen zugänglich zu machen, die mit der Anerkennung des Praxissemesters oder der Abschlussarbeit befasst sind.

\newpage

% Inhaltsverzeichnis

\tableofcontents

\newpage

% Abkürzungsverzeichnis, alphabetisch geordnet

\section*{Abkürzungsverzeichnis}
\label{sec:abkuerzungsverzeichnis}
\addcontentsline{toc}{section}{\nameref{sec:abkuerzungsverzeichnis}}

\begin{table}[!h]
\begin{tabular}{l l}
Abk.1 & Abkürzung 1 \\
Abk.2 & Abkürzung 2 \\
Abk.3 & Abkürzung 3 \\
\end{tabular}
\end{table}

\newpage

% Abbildungsverzeichnis

% NICHT LÖSCHEN. Entfernt eine redundante Überschrift des Abbildungsverzeichnisses und schreibt sie neu.
\makeatletter
\renewcommand\listoffigures{%
        \@starttoc{lof}%
}
\makeatother

\section*{Abbildungsverzeichnis}
\label{sec:abbildungsverzeichnis}
\addcontentsline{toc}{section}{\nameref{sec:abbildungsverzeichnis}}

\listoffigures

\newpage

% Symbolverzeichnis

\begin{comment}
\section*{Symbolverzeichnis}
\label{sec:symbolverzeichnis}
\addcontentsline{toc}{section}{\nameref{sec:symbolverzeichnis}}

\newpage
\end{comment}

% Tabellenverzeichnis

% NICHT LÖSCHEN. Entfernt eine redundante Überschrift des Tabellenverzeichnisses und schreibt sie neu.

\begin{comment}
\makeatletter
\renewcommand\listoftables{%
        \@starttoc{lot}%
}
\makeatother

\section*{Tabellenverzeichnis}
\label{sec:tabellenverzeichnis}
\addcontentsline{toc}{section}{\nameref{sec:tabellenverzeichnis}}

\listoftables

\newpage
\end{comment}

% Codelistingverzeichnis

\begin{comment}
\section*{Codelistingverzeichnis}
\label{sec:codeverzeichnis}
\addcontentsline{toc}{section}{\nameref{sec:codeverzeichnis}}

\newpage
\end{comment}

\section{Kapitel A}

\Blindtext

\blindtext

\blindtext

\Blindtext

\newpage

\section{Kapitel B}

\Blindtext

\blindtext

\blindtext

\Blindtext

\newpage

\section{Kapitel C}

\Blindtext

\blindtext

\blindtext

\Blindtext

\newpage

\section{Ein anderes Kapitel}

Die Quellen \cite{bi:book1}, \cite{bi:book2} sind wissenschaftlich fundiert \cite{bi:book3}.

\newpage

\section*{Literaturverzeichnis}
\label{sec:literaturverzeichnis}
\addcontentsline{toc}{section}{\nameref{sec:literaturverzeichnis}}

\begingroup
\renewcommand{\section}[2]{}%
\bibliography{quellenverzeichnis}
\endgroup

\end{document}